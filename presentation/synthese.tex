\documentclass{beamer}

\mode<presentation>
{
  \usetheme{Singapore}
  \usecolortheme{default}
  \usefonttheme{default}
  \setbeamertemplate{navigation symbols}{}
  \setbeamertemplate{caption}[numbered]
  \setbeamertemplate{footline}[frame number]
}

\usepackage[frenchb]{babel}
\usepackage[T1]{fontenc}
\usepackage[utf8]{inputenc}
\usepackage{tikz-qtree}
\usepackage[thinlines]{easytable}
\usetikzlibrary{arrows}

\title{Synthèse de mi parcours}
\author{Thibault de Boutray, Louis Vignier}
\institute{CentraleSupélec}

\AtBeginSection[]
{
  \begin{frame}
  \frametitle{Sommaire}
  \tableofcontents[currentsection]
  \end{frame}
}

\begin{document}

\begin{frame}
  \titlepage
\end{frame}

\begin{frame}{Plan}
  \tableofcontents
\end{frame}

\section{Présentation du sujet}

\subsection{Problématique}

\begin{frame}{Problématique}
 Comment vérifier la présence effective d'un utilisateur à un évènement, et dans la durée ? 
\end{frame}


\subsection{Cahier des charges}

\begin{frame}{Assurer la présence d’un utilisateur à un événement}	
    \begin{itemize}
        \item Vérification ponctuelle\pause
        \item Vérification sur la durée complète de l'évènement
    \end{itemize}
\end{frame}

\begin{frame}{Propriétés d'utilisation}
    \begin{itemize}
        \item Simple d’utilisation pour les maîtres de conférence\pause
        \item Supporte ~150 participants\pause
        \item Peu d’actions de la part des participants\pause
        \item Doit pouvoir fonctionner sur une journée entière\pause
        \item Assurer la présence effective et unique de l’utilisateur
    \end{itemize}
\end{frame}

\begin{frame}{Propriétés de Privacy}
    \begin{itemize}
        \item Eviter le “flicage” (par ex. suivi non sollicité en arrière-plan)\pause
        \item Respecter la réglementation en vigueur (RGPD)\pause
        \item Avoir un mode “Anonyme” avec un simple compte pour réaliser des statistiques (choisi par le MDC)
    \end{itemize}
\end{frame}


\section{Les attaques}

\section{Protocoles délimiteurs de distance}

\subsection{Présentation}

\subsection{Attaques}

\section{Problèmes technologiques}

\section{Swiss-Knife}

\end{document}
