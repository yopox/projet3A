\documentclass{beamer}

\mode<presentation>
{
  \usetheme{Singapore}
  \usecolortheme{default}
  \usefonttheme{default}
  \setbeamertemplate{navigation symbols}{}
  \setbeamertemplate{caption}[numbered]
  \setbeamertemplate{footline}[frame number]
}

\usepackage[frenchb]{babel}
\usepackage[T1]{fontenc}
\usepackage[utf8]{inputenc}
\usepackage{tikz-qtree}
\usepackage[thinlines]{easytable}
\usetikzlibrary{arrows}

\title{Synthèse de mi parcours}
\author{Thibault de Boutray, Louis Vignier}
\institute{CentraleSupélec}

\AtBeginSection[]
{
  \begin{frame}
  \frametitle{Sommaire}
  \tableofcontents[currentsection]
  \end{frame}
}

\begin{document}

\begin{frame}
  \titlepage
\end{frame}

\begin{frame}{Plan}
  \tableofcontents
\end{frame}

\section{Présentation du sujet}

\begin{frame}
    
\end{frame}

\subsection{Problématique}

\begin{frame}{Problématique}
 Comment vérifier la présence effective d'un utilisateur à un évènement, et dans la durée ? 
\end{frame}


\subsection{Cahier des charges}

\begin{frame}{Assurer la présence d’un utilisateur à un événement}	
    \begin{itemize}
        \item Vérification ponctuelle\pause
        \item Vérification sur la durée complète de l'évènement
    \end{itemize}
\end{frame}

\begin{frame}{Propriétés d'utilisation}
    \begin{itemize}
        \item Simple d’utilisation pour les maîtres de conférence\pause
        \item Supporte ~150 participants\pause
        \item Peu d’actions de la part des participants\pause
        \item Doit pouvoir fonctionner sur une journée entière\pause
        \item Assurer la présence effective et unique de l’utilisateur
    \end{itemize}
\end{frame}

\begin{frame}{Propriétés de Privacy}
    \begin{itemize}
        \item Eviter le “flicage” (par ex. suivi non sollicité en arrière-plan)\pause
        \item Respecter la réglementation en vigueur (RGPD)\pause
        \item Avoir un mode “Anonyme” avec un simple compte pour réaliser des statistiques (choisi par le MDC)
    \end{itemize}
\end{frame}


\section{Les attaques}

\begin{frame}
    
\begin{center}
\begin{tikzpicture}
    % Amphi
    \draw [black, thick, dotted] (0,0) circle [radius=3];
    \node [above] at (0,3) {Amphi};

    % Prof
    \draw [fill] (0,-0.2) circle [radius=0.05];
    \node [above] at (0,-0.2) {Prof};

    \pause

    % Eleve seul
    \draw [fill, gray] (4,0) circle [radius=0.05];
    \node [right, gray] at (4,0) {Élève seul};
    \draw [->, thick, line width=1.2, gray] (3.75,0) -- (0.5,0);

    \pause

    % Complicité interne
    \draw [fill, red] (1.3, -1.3) circle [radius=0.05];
    \node [above right, red] at (1.3, -1.3) {BP};
    \draw [fill, red] (3,-3) circle [radius=0.05];
    \node [right, red] at (3,-3) {Complicité interne};
    \draw [->, thick, dashed, line width=1.2, red] (2.85,-2.85) -- (1.45,-1.45);
    \draw [->, thick, line width=1.2, red] (1.15,-1.15) -- (0.35,-0.35);

    \pause

    % Sans complicité
    \draw [fill, orange] (1.3, 1.3) circle [radius=0.05];
    \node [below right, orange] at (1.3, 1.3) {A};
    \draw [fill, orange] (3,3) circle [radius=0.05];
    \node [right, orange] at (3,3) {Sans complicité};
    \draw [->, thick, dashed, line width=1.2, orange] (2.85,2.85) -- (1.45,1.45);
    \draw [->, thick, line width=1.2, orange] (1.15,1.15) -- (0.35,0.35);

    \pause

    % Pokemon
    \draw [cyan, thick] (-1.3,-1.3) circle [radius=0.5];
    \node [right, cyan] at (-0.8, -1.3) {BP};
    \draw [fill, cyan] (-1.1,-1.3) circle [radius=0.05];
    \draw [fill, cyan] (-1.3,-1.3) circle [radius=0.05];
    \draw [fill, cyan] (-1.5,-1.3) circle [radius=0.05];
    \draw [fill, cyan] (-1.3,-1.5) circle [radius=0.05];
    \draw [fill, cyan] (-1.3,-1.1) circle [radius=0.05];
    \draw [->, thick, line width=1.2, cyan] (-0.94,-0.94) -- (-0.35,-0.35);

    \pause

    % sybil
    \draw [fill, olive] (-1.3,1.3) circle [radius=0.05];
    \node [above left, olive] at (-1.3, 1.3) {BP};
    \draw [->, thick, line width=1.2, olive] (-1.15,1.35) to [out=15,in=105] (-0.15,0.45);
    \draw [->, thick, line width=1.2, olive] (-1.35,1.15) to [out=255,in=165] (-0.45,0.15);

\end{tikzpicture}
\end{center}

\end{frame}

\section{Protocoles délimiteurs de distance}

\subsection{Présentation}

\subsection{Attaques}

\section{Problèmes technologiques}

\section{Swiss-Knife}

\end{document}
